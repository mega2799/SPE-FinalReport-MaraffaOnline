\chapter{Introduction}
\label{ch:into} % This how you label a chapter and the key (e.g., ch:into) will be used to refer this chapter ``Introduction'' later in the report. 
% the key ``ch:into'' can be used with command \ref{ch:intor} to refere this Chapter.
MaraffaOnline è un'applicazione che permette alle persone di giocare al gioco di carte \href{https://it.wikipedia.org/wiki/Marafone_Beccacino}{Maraffa/Beccacino}. 
Il progetto consiste nell’eseguire una manutenzione evolutiva del gioco di carte reperibile su MaraffaOnline, attualmente sviluppato dalla prof.ssa Lumini.
In particolare la nuova versione avrà un'architettura a microservizi e introdurrà anche nuove funzionalità come formazione personalizzata delle squadre, una nuova modalità 
di gioco (vittoria 11 a 0 in caso di violazione delle regole da parte di una squadra), salvataggio delle statistiche delle partite e degli utenti, ...
\\
Per lo sviluppo è stato seguito un approccio Domain Driven Design, per il quale si è approfondito il dominio del gioco.
\\
È stata posta particolare attenzione alle tecniche di continuos integration, alle quali è stato dedicato uno dei capitoli di questo report.

% Meeting people in real life to challenge them in games can be sometimes difficult, especially when someone moves into a different city or a new disease force us to stay at home. 
% MaraffaOnline is a card game that allow players from different location to play together spending their time. Users can meet, create a game and play their cards using only allowed word and have some fun!
% The project consists of carrying out an evolutionary maintenance of the card game [Maraffa/Beccacino](https://it.wikipedia.org/wiki/Marafone_Beccacino) available on [MaraffaOnline](https://www.maraffaonline.it/), currently developed by Prof. Lumini. There is actually  a frontend in Angular, a basic API in C# and the user management is absent. Our work will be to add new functionalities and perform refactoring not only at the code level, but also at the architectural level.


